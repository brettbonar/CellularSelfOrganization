Self-organization could be defined as the undirected movement of a chaotic structure into an ordered one. In a biological system, this may be seen as a formation of cells gradually reforms into a regular arrangement. Such arrangements could be lattices or grids that are identified as having regular or repeating patterns. Determining the behaviors with which to compose a self-organizing system is a nontrivial task. While many such systems have already been identified, the considerable amount of traits comprising cellular behaviors would make an exhaustive search of undiscovered systems impossible. The ability to discover these types of systems could prove valuable to disciplines such as medicine and bioengineering where regular patterns have real applications.